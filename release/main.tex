\documentclass[
  lang=cn, % 目前只支持中文
  degree=bachelor, % 目前只支持学士论文(设计)
  type=paper, % 支持paper(论文)或design(设计)
  %openany,oneside % 单面打印
  openright,blankleft,twoside % 双面打印
]{nwafuthesis}

% 载入需要的宏包
% 请按自己的论文排版需求,加载需要的宏包

\usepackage{subfig}
\usepackage{rotating}
\usepackage[usenames,dvipsnames]{xcolor}
\usepackage{tikz}
\usepackage{pgfplots}
\pgfplotsset{compat=1.16}
\pgfplotsset{
  table/search path={./figs/},
}
\usepackage{ifthen}
\usepackage{longtable}
\usepackage{siunitx}
\usepackage{listings}
\usepackage{multirow}
\usepackage[bottom]{footmisc}
\usepackage{pifont}

\lstdefinestyle{lstStyleBase}{%
  basicstyle=\small\ttfamily,
  aboveskip=\medskipamount,
  belowskip=\medskipamount,
  lineskip=0pt,
  boxpos=c,
  showlines=false,
  extendedchars=true,
  upquote=true,
  tabsize=2,
  showtabs=false,
  showspaces=false,
  showstringspaces=false,
  numbers=left,
  numberstyle=\footnotesize,
  linewidth=\linewidth,
  xleftmargin=\parindent,
  xrightmargin=0pt,
  resetmargins=false,
  breaklines=true,
  breakatwhitespace=false,
  breakindent=0pt,
  breakautoindent=true,
  columns=flexible,
  keepspaces=true,
  framesep=3pt,
  rulesep=2pt,
  framerule=1pt,
  backgroundcolor=\color{gray!5},
  stringstyle=\color{green!40!black!100},
  keywordstyle=\bfseries\color{blue!50!black},
  commentstyle=\slshape\color{black!60}}

\usepackage{texboxie}


%%% Local Variables: 
%%% mode: latex
%%% TeX-master: "../main.tex"
%%% End:

% 进行必要的设置
\iffalse
  % 本块代码被上方的 iffalse 注释掉,如需使用,请改为 iftrue
  % 使用 Noto 字体替换中文宋体、黑体
  \setCJKfamilyfont{\CJKrmdefault}[BoldFont=Noto Serif CJK SC Bold]{Noto Serif CJK SC}
  \renewcommand\songti{\CJKfamily{\CJKrmdefault}}
  \setCJKfamilyfont{\CJKsfdefault}[BoldFont=Noto Sans CJK SC Bold]{Noto Sans CJK SC Medium}
  \renewcommand\heiti{\CJKfamily{\CJKsfdefault}}
\fi

% ==============LaTeX命令排版命令========================
\newcommand\cs[1]{\texttt{\textbackslash#1}}
%\newcommand\pkg[1]{\texttt{#1}\textsuperscript{PKG}}
%\newcommand\env[1]{\texttt{#1}}
\newcommand{\note}[1]{{%
  \color{magenta}{\bfseries 注意:}\emph{#1}}}

% ==============自定义定理环境========================
\theoremstyle{nwafuplain}
\nwafutheoremchapu{definition}{定义}
\nwafutheoremchapu{assumption}{假设}
\nwafutheoremchap{exercise}{练习}
\nwafutheoremchap{nonsense}{胡诌}
\nwafutheoremg[句]{lines}{句子}

% 设置插图目录
\graphicspath{{./figs/},{./logo/}}

% 重定义强调字体的代码
% 在此设置为加粗,注意需要使用etoolbox宏包
\makeatletter
\let\origemph\emph
\newcommand*\emphfont{\normalfont\bfseries}
\DeclareTextFontCommand\@textemph{\emphfont}
\newcommand\textem[1]{%
  \ifdefstrequal{\f@series}{\bfdefault}
    {\@textemph{\CTEXunderline{#1}}}
    {\@textemph{#1}}%
}
\RenewDocumentCommand\emph{s o m}{%
  \IfBooleanTF{#1}
    {\textem{#3}}
    {\IfNoValueTF{#2}
      {\textem{#3}\index{#3}}
      {\textem{#3}\index{#2}}%
     }%
}
\makeatother   

%% 自定义相关的名称宏命令
%% ==================================================
%% \newcommand{\yourcommand}[参数个数]{内容}
% 西北农林科技大学各单位名称
\newcommand{\nwsuaf}{西北农林科技大学}
\newcommand{\cie}{信息工程学院}
\newcommand{\ca}{农学院}
\newcommand{\cpp}{植物保护学院}
\newcommand{\ch}{园艺学院}
\newcommand{\cast}{动物科技学院}
\newcommand{\cvm}{动物医学院}
\newcommand{\cf}{林学院}
\newcommand{\claa}{风景园林艺术学院}
\newcommand{\cnre}{资源环境学院}
\newcommand{\cwrae}{水利与建筑工程学院}
\newcommand{\cmee}{机械与电子工程学院}
\newcommand{\cfse}{食品科学与工程学院}
\newcommand{\ce}{葡萄酒学院}
%\newcommand{\cls}{生命科学学院}
\newcommand{\cst}{理学院}
\newcommand{\ccp}{化学与药学院}
\newcommand{\cem}{经济管理学院}
\newcommand{\cm}{马克思主义学院}
\newcommand{\dfl}{外语系}
\newcommand{\iec}{创新实验学院}
\newcommand{\ci}{国际学院}
\newcommand{\dpe}{体育部}
\newcommand{\cvae}{成人教育}
\newcommand{\iswc}{水土保持研究所}

% ==============自定义的双引号、字体强调等命令========================
% 定义提醒字体
\newcommand{\alert}[1]{\textcolor{red}{\textbf{#1}}}
% 定义引号命令
\newcommand{\qtmark}[1]{{\symbol{"201C}}#1{\symbol{"201D}}}
%\newcommand{\qtmark}[1]{``#1''}%``''
% 定义带引号的加粗强调命令
\newcommand{\qtb}[1]{\qtmark{\emph{#1}}}
% 定义带引号的加粗加红强调命令
\newcommand{\qtbr}[1]{\qtmark{\emph{\alert{#1}}}}


%%% Local Variables: 
%%% mode: latex
%%% TeX-master: "../main.tex"
%%% End:


% 设置文档基本信息,\linebreak 前面不要有空格,否则在无需换行的场合,中文之间的空格无法消除
\nwafuset{
  title = {\nwafuthesis{} 快速上手\linebreak 示例文档}, % 论文题目
  author = {\LaTeX{}er},                              % 论文作者姓名
  college = {信息工程学院},                             % 学院名称 
  advisers = {耿楠},                                   % 指导教师姓名
  coadvisers = {Donald Knuth\quad 大师, tex.se 大牛们}, % 协助指导教师姓名
  gradyear = {2019},                                   % 毕业届数(年)
  % applydate = {二〇一八年六月}                         % 完成日期(默认为当前日期)
  %
  % 本科
  major = {计算机科学与技术},                            % 专业
  studentid = {2015012821},                            % 学号
  classid = {152},                                     % 班级号,请勿有其它内容
}

% 英文(本科只需要设置英文题目)
\nwafusetEn{
  title = {\nwafuthesis{} Quick Start\linebreak and Document Snippets},% 论文英文题目
  author = {\LaTeX{}er},
  college = {College of Infomation Engineering},
  majorsubject = {Computer Science},
  advisers = {Geng Nan}
}

% biblatex宏包的参考文献数据源
\addbibresource[location=local]{bib/sample.bib}

\begin{document}

% 排版封页
\makecover
% 排版独创性和版权声明页
\makedeclare
% 排版摘要
% 本文件是示例论文的一部分
% 论文的主文件位于上级目录的 `main.tex`

% 中文摘要
\begin{abstract}
本文介绍如何使用\nwafuthesis{} 文档类撰写西北农林科技大学学位论文。

首先介绍如何获取并编译本文档,然后展示论文部件的实例,最后列举部分常用宏包的使用方法。
\end{abstract}
% 中文关键词(用英文","分割)
\keywords{学位论文, 模板, \nwafuthesis}

% 英文摘要
\begin{abstractEn}
This document introduces \nwafuthesis, the \LaTeX{} document class for NUAA Thesis.

First, we show how to get the source code and compile this document.
Then we provide snippets for figures, tables, equations, etc.
Finally we enforce some usage patterns.
\end{abstractEn}
% 英文文关键词(用英文","分割)
\keywordsEn{NWAFU thesis, document class, space is accepted here}

%%% Local Variables: 
%%% mode: latex
%%% TeX-master: "../main.tex"
%%% End:

\makeabstract

% 排版目录
\frontmatter
% 如果需要调整目录层级数量的话,取消下一行注释,数字含义: 0=chapter, 1=section, 2=subsection
% \setcounter{tocdepth}{1}
\expandafter\nwafutableofcontents

% 排版正文
\mainmatter
% 本文件是示例论文的一部分
% 论文的主文件位于上级目录的 `main.tex`

\chapter{快速上手}

\section{欢迎}

欢迎使用 \nwafuthesis,本文档将介绍如何利用 \nwafuthesis 模板进行学位论文写作,
假设读者有 \LaTeX 写作经验,并会使用搜索引擎解决常见问题。

源代码托管于 \url{https://github.com/registor/nwafuthesis},
欢迎来提 issue/PR。

\section{\LaTeX 环境准备}

由于本模板使用了大量宏包,因此对 \LaTeX 环境有不少要求。
推荐使用以下打 \ding{51} 的 \LaTeX 发行版:
\begin{itemize}
\item[\ding{51}]\TeX~Live 请安装以下 collection:langchinese, latexextra, science, pictures, fontsextra;\\
如果觉得安装体积太大的话,可以看 \texttt{.ci/texlive.pkgs} 列出的所需宏包;
\item[\ding{51}]MiK\TeX 可能国内镜像服务器无法联通,如果无法联通,建议隔天再试; \\
因为 MiK\TeX{} 能自动下载安装宏包,推荐 Windows 用户使用。
\item[\ding{53}]CTeX  \qtbr{不推荐},可能会有宏包缺失、版本过旧导致无法编译现象。
\end{itemize}

\section{编译模板和文档}

只有在找不到 \verb|nwafuthesis.cls| 文件的时候,才需要执行本步骤。

进入模板的根目录,运行 \verb|build.bat|(Windows) 或 \verb|build.sh|(其他系统),
它会生成模板 \verb|nwafuthesis.cls| 以及对应的说明文档 \verb|nwafuthesis.pdf|。

\section{使用模板}

论文写作时,请确认\textbf{论文的目录}(\verb|main.tex|所在的目录)下有以下文件:
\begin{itemize}
  \item \verb|nwafuthesis.cls| 文档模板;
  \item \verb|logo/| 文件夹,内含学校的LOGO图标;
\end{itemize}

如果论文目录下没有这些文件的话,请从本模板根目录复制一份。

\section{开始写作}

最方便的开始方法,莫过于修改现有的文稿。因此推荐直接修改本文档,建议按
如下结构组织和管理写作过程中的各个文件:
\begin{itemize}
  \item \verb|main.tex| 主文件,定义了文档包含的内容,建议不要随意更改
    \verb|main.tex|文件名,其它的\verb|*.tex|中会用到该文件名信息;
  \item \verb|setup/packages.tex| 载入需要的宏包,可根据需要进行增
    加或删除;
  \item \verb|setup/format.tex| 全局使用的自定义命令、相关设置等;  
  \item \verb|content/*.tex| 文件夹,按章节拆分的文档内容,分章节以存
    放在这里;
  \item \verb|figs/| 文件夹,插图文件;  
  \item \verb|bib/| 文件夹,内含参考文献数据库,文献数据库是纯文本文
    件,请务必采用\qtbr{UTF8编码}存储;
  \item \verb|ref/| 文件夹,可有可无,内含写作过程中用到的资料、参考文
    献、记录等;  
\end{itemize}

完成部分或所有\verb|*.tex|撰写和修改后,可以在命令行使用 \verb|latexmk -xelatex main|
进行编译输出\verb|main.pdf|文件,可以根据需要对结果\texttt{pdf}文件进行改名。

也可以使用\texttt{TeXstudio}、\texttt{vscode}等图形界面的编辑器的进行
编译输出。

\section{打印论文}

如果论文需要双面打印的话,请务必修改文档类选项,编译双面打印用的 PDF 文件。
具体地说,在主文件的头部,去除 \texttt{openany, oneside},改成 \texttt{openright, blankleft, twoside}。

%%% Local Variables: 
%%% mode: latex
%%% TeX-master: "../main.tex"
%%% End:

% 本文件是示例论文的一部分
% 论文的主文件位于上级目录的 `main.tex`

\chapter{使用示例}

本章介绍一些宏包的常用方法,希望能为读者写作时提供参考。

\section{插图}

首先讨论一下插图的格式,在 \LaTeX{} 环境下,
\begin{enumerate}
\item 推荐使用宏包来绘制插图,如 \pkg{tikz},它兼容所有 \LaTeX{} 环境,
字体能与全文统一,质量最佳,但是需要的学习成本较大。
请务必先阅读 \pkg{tikz} 文档教程,
然后可以去 texample\footnote{\url{http://texample.net/tikz}} 等网站上找类似的例子,
也可以使用 GeoGebra\footnote{\url{https://www.geogebra.org}} 之类的工具来生成\TeX 代码,
效果可以参见\autoref{fig:tikzrot};
\item 其次推荐使用其他绘图工具生成的 \verb|PDF|、 \verb|EPS| 格式的\qtbr{矢量图},
\verb|svg| 格式可以通过 inkscape 软件转换成带 \TeX{}文本代码的 \verb|PDF|。效果可以参见\autoref{fig:logo};
\item 诚然,\verb|PNG|、 \verb|jpeg| 之类的位图也能做插图,不过
  \qtbr{质量堪忧},小心导师狠批;
\item 最后,一般论文都是\emph{单色印刷}的,请确保插图在黑白打印情况下的清晰度。
\end{enumerate}

\begin{figure}[htb]
  \newcounter{density}
  \setcounter{density}{20}
  \input{figs/tikz_rot}
  \caption{tikz例子}
  \label{fig:tikzrot}
\end{figure}

\begin{figure}[htb]
  \includegraphics[width=3cm]{nwafu-circle}
  \caption{一个校徽}
  \label{fig:logo}
\end{figure}

\subsection{\env{figure}插图浮动体}

为避免由于大小变化时,当前页面没有足够空间排插图而造成的
\emph{页面留白}问题,强烈建议使用\env{figure}环境排版插图,
同时,\env{figure}环境还能够实现插图编号及交叉引用的自动化,如\autoref{fig:tikzrot}所示。

\subsection{排版子图}

如多个子图共用题注,需加载额外宏包,可以使
用\pkg{floatrow}、\pkg{subcaption} 或 \pkg{subfig},注
意\pkg{subcaption} 和 \pkg{subfig}两个宏包是互斥的。另
外,\pkg{subcaption} 貌似与 \pkg{geometry} 有些冲突,会导致多行的图表的
最后一行无法居中,而 \pkg{geometry} 是设置页边距的必用宏包。所以个人推
荐使用 \pkg{floatrow}或\pkg{subfig},效果可以参考\autoref{fig:sub2}。

\begin{figure}[htb]
  \subfloat[左边的大校徽\label{fig:sub1}]{\includegraphics[width=3cm]{nwafu-circle}}\quad
  \subfloat[短标题:小校徽][小校徽,题注很长,不过请各位放心,它会自动换行\label{fig:sub2}]
  {\includegraphics[width=3.2cm]{nwafu-bar}}
  \caption{包含两张图片的插图}
  \label{fig:subfigs}
\end{figure}

如果需要插入图表的话,可以考虑使用 \pkg{pgfplots} 宏包,效果参见\autoref{fig:plots};
也可以用 Matplotlib、MatLab、Mathematica 之类的工具导出成兼容格式的图片。

\begin{figure}[htb]
  \subfloat[二维图像\label{fig:func}]{\input{figs/plot_2d}} \quad
  \subfloat[三维图像\label{fig:sum}]{\input{figs/plot_3d}}
  \caption{利用 \pkg{pgfplot} 绘制图表}
  \label{fig:plots}
\end{figure}

如果真的需要让十几张图片共用一个题注的话,
需要手工拆分成多个 \env{float} 并用 \cs{ContinuedFloat} 来拼接,
不过直接多次使用 \cs{caption} 会在图表清单里产生多个重复条目,需要一点点小技巧
(设置图表目录标题为空)。
建议将浮动位置指定为 \verb|t|,以确保分散至多页的图能占用整个页面,手工分页才能靠谱。
效果可以参见\autoref{fig:subfigss} 的\autoref{fig:logo6}。

\begin{figure}[t]
  \subfloat[校徽$\times 1$]{\includegraphics[width=2cm]{nwafu-circle}}\quad
  \subfloat[校徽$\times 2$]{\includegraphics[width=2cm]{nwafu-circle}}\\
  \subfloat[校徽$\times 3$]{\includegraphics[width=2cm]{nwafu-circle}}\quad
  \subfloat[校徽$\times 4$]{\includegraphics[width=2cm]{nwafu-circle}}
  \caption{包含多张图片的插图}
  \label{fig:subfigss}
\end{figure}
\begin{figure}[t]
  \ContinuedFloat
  \subfloat[校徽$\times 5$]{\includegraphics[width=2cm]{nwafu-circle}}\quad
  \subfloat[校徽$\times 6$ \label{fig:logo6}]{\includegraphics[width=2cm]{nwafu-circle}}\\
  \subfloat[校徽$\times 7$]{\includegraphics[width=2cm]{nwafu-circle}}\quad
  \subfloat[校徽$\times 8$]{\includegraphics[width=2cm]{nwafu-circle}}
  % 指定图表清单中的标题为[],即可将其消除,避免目录中出现重复条目
  \caption[]{包含多张图片的插图(续)}
\end{figure}

如果需要插入一张很大的图片的话,可以使用 \pkg{rotating} 提供的 \env{sidewaysfigure},
它能将插图放置在单独的页面上,如果文档使用 \verb|twoside| 选项的话,它会根据页面方向,
设置 \ang{90} 或 \ang{270} 旋转,可能需要编译两遍才能设置正确的旋转方向。
不过可能有一个问题,\env{sidewaysfigure} 中使用 \cs{subfloat} 可能无法准确标号,
需要手工重置 \texttt{subfigure} 计数器。
效果参见\autoref{fig:fullpage1} 和\autoref{fig:fullpage2}。

\setcounter{subfigure}{0}
\begin{sidewaysfigure}
  \subfloat[First caption\label{fig:fp1}]{\includegraphics[width=.8\textheight]{nwafu-bar}} \\
  \subfloat[Second caption]{\includegraphics[height=2cm]{nwafu-bar}}
  \caption{一幅占用完整页面的图片}
  \label{fig:fullpage1}
\end{sidewaysfigure}

\setcounter{subfigure}{0}
\begin{sidewaysfigure}
  \subfloat[First caption]{\includegraphics[height=2cm]{nwafu-bar}} \\
  \subfloat[Second caption]{\includegraphics[width=.8\textheight]{nwafu-bar}}
  \caption{又一幅占用完整页面的图片}
  \label{fig:fullpage2}
\end{sidewaysfigure}

\section{表格}

由于封面页,本模板预先加载了 \pkg{array} 和 \pkg{tabu},如果需要其他表格的宏包,
请自行加载。

如果需要插入一个简易的表格,可以只使用 \env{tabular} 环境,如\autoref{tab:city}。
\begin{table}[htb]
  \caption[城市人口]{城市人口数量排名 (source: Wikipedia)\label{tab:city}}
  \begin{tabular}{lr}
    \toprule
    城市 & 人口 \\
    \midrule
    Mexico City & 20,116,842\\
    Shanghai & 19,210,000\\
    Peking & 15,796,450\\
    Istanbul & 14,160,467\\
    \bottomrule
  \end{tabular}
\end{table}

也可以使用 \env{tabu} 环境,它可以更灵活地设置列宽,但它有一些 bug,如\autoref{tab:tabu}。
\begin{table}[htb]
  \caption{\env{tabu} 注意事项 \label{tab:tabu}}
  \begin{tabu} to .9\textwidth {XX[2]<{\strut}} \toprule
    默认列 & 有修正的列 \\ \midrule
    \env{tabu} 的 bug? \par This line is BAD & 注意左侧最后一行后的垂直空格 \\ \midrule
    注意对比最后一行 &
      bug 会影响多行的 \env{tabu} 表格 \par
      bug 的修正方法是在段落后面加 \cs{strut} \par
      This line is Good \\ \midrule
    垂直居中没效果 & 改用 \env{tabular} \\ \midrule
    与新版 \pkg{array} 不兼容 & 谨慎使用,切勿用 \texttt{tabu spread} \\ \bottomrule
  \end{tabu}
\end{table}

如果需要对某一列的小数点对齐,或者带有单位,或者需要做四舍五入的处理,可以尝试配合 \pkg{siunitx} 一起使用。
非常推荐看一下 \pkg{siunitx} 文档的,至少看一下\qtmark{Hints for using siunitx}一节的输出结果,
\autoref{tab:xmpl:mixed} 来自于该文档的 7.14 节。

\begin{table}[htb]
  \caption{Tables where numbers have different units}
  \label{tab:xmpl:mixed}
  \begin{tabular}
    {
      >{$}l<{$}
      S[table-format = 2.3(1)]
      S[table-format = 3.3(1)]
    }
    \toprule
      & {One} & {Two} \\
    \midrule
    a / \si{\angstrom}   &  1.234(2) &   5.678(4) \\
    \beta / \si{\degree} & 90.34(4)  & 104.45(5)  \\
    \mu / \si{\per\mm}   &  0.532    &   0.894    \\
    \bottomrule
  \end{tabular}
  \hfil
  \begin{tabular}
    {S[table-format=1.3]@{\,}s[table-unit-alignment = left]}
    \toprule
    \multicolumn{2}{c}{Heading} \\
    \midrule
    1.234 & \metre   \\
    0.835 & \candela \\
    4.23  & \joule\per\mole \\
    \bottomrule
  \end{tabular}
\end{table}

如果表格内容很多,导致无法放在一页内的话,需要用 \env{longtable} 或 \env{longtabu} 进行分页。
\autoref{tab:performance} 是来自 \nuaathesis{} 的一个长表格的例子。

\begin{longtable}[c]{c*{6}{r}}
        \caption[实验数据]{实验数据,这个题注十分的长,注意这在索引中的处理方式,还有 \cs{caption} 后面的双反斜杠}\label{tab:performance}\\
        \toprule
        \multirow{2}{*}{测试程序} & \multicolumn{1}{c}{正常运行} & \multicolumn{1}{c}{同步} & \multicolumn{1}{c}{检查点} & \multicolumn{1}{c}{卷回恢复}
        & \multicolumn{1}{c}{进程迁移} & \multicolumn{1}{c}{检查点} \\
        & \multicolumn{1}{c}{时间 (s)}& \multicolumn{1}{c}{时间 (s)}&
        \multicolumn{1}{c}{时间 (s)}& \multicolumn{1}{c}{时间 (s)}& \multicolumn{1}{c}{时间 (s)}& \multicolumn{1}{c}{文件 (KB)} \\ \midrule
        \endfirsthead
        \multicolumn{7}{c}{\nwafufontcaption 续表~\thetable\hskip1em 实验数据}\\
        \toprule
        \multirow{2}{*}{测试程序} & \multicolumn{1}{c}{正常运行} & \multicolumn{1}{c}{同步} & \multicolumn{1}{c}{检查点} & \multicolumn{1}{c}{卷回恢复}
        & \multicolumn{1}{c}{进程迁移} & \multicolumn{1}{c}{检查点} \\
        & \multicolumn{1}{c}{时间 (s)}& \multicolumn{1}{c}{时间 (s)}&
        \multicolumn{1}{c}{时间 (s)}& \multicolumn{1}{c}{时间 (s)}& \multicolumn{1}{c}{时间 (s)}& \multicolumn{1}{c}{文件(KB)} \\ \midrule
        \endhead
        \hline
        \multicolumn{7}{r}{续下页}
        \endfoot
        \endlastfoot
        CG.A.2 & 23.05 & 0.002 & 0.116 & 0.035 & 0.589 & 32491 \\
        CG.A.4 & 15.06 & 0.003 & 0.067 & 0.021 & 0.351 & 18211 \\
        CG.A.8 & 13.38 & 0.004 & 0.072 & 0.023 & 0.210 & 9890 \\
        CG.B.2 & 867.45 & 0.002 & 0.864 & 0.232 & 3.256 & 228562 \\
        CG.B.4 & 501.61 & 0.003 & 0.438 & 0.136 & 2.075 & 123862 \\
        CG.B.8 & 384.65 & 0.004 & 0.457 & 0.108 & 1.235 & 63777 \\
        MG.A.2 & 112.27 & 0.002 & 0.846 & 0.237 & 3.930 & 236473 \\
        MG.A.4 & 59.84 & 0.003 & 0.442 & 0.128 & 2.070 & 123875 \\
        MG.A.8 & 31.38 & 0.003 & 0.476 & 0.114 & 1.041 & 60627 \\
        MG.B.2 & 526.28 & 0.002 & 0.821 & 0.238 & 4.176 & 236635 \\
        MG.B.4 & 280.11 & 0.003 & 0.432 & 0.130 & 1.706 & 123793 \\
        MG.B.8 & 148.29 & 0.003 & 0.442 & 0.116 & 0.893 & 60600 \\
        LU.A.2 & 2116.54 & 0.002 & 0.110 & 0.030 & 0.532 & 28754 \\
        LU.A.4 & 1102.50 & 0.002 & 0.069 & 0.017 & 0.255 & 14915 \\
        LU.A.8 & 574.47 & 0.003 & 0.067 & 0.016 & 0.192 & 8655 \\
        LU.B.2 & 9712.87 & 0.002 & 0.357 & 0.104 & 1.734 & 101975 \\
        LU.B.4 & 4757.80 & 0.003 & 0.190 & 0.056 & 0.808 & 53522 \\
        LU.B.8 & 2444.05 & 0.004 & 0.222 & 0.057 & 0.548 & 30134 \\
        CG.B.2 & 867.45 & 0.002 & 0.864 & 0.232 & 3.256 & 228562 \\
        CG.B.4 & 501.61 & 0.003 & 0.438 & 0.136 & 2.075 & 123862 \\
        CG.B.8 & 384.65 & 0.004 & 0.457 & 0.108 & 1.235 & 63777 \\
        MG.A.2 & 112.27 & 0.002 & 0.846 & 0.237 & 3.930 & 236473 \\
        MG.A.4 & 59.84 & 0.003 & 0.442 & 0.128 & 2.070 & 123875 \\
        MG.A.8 & 31.38 & 0.003 & 0.476 & 0.114 & 1.041 & 60627 \\
        MG.B.2 & 526.28 & 0.002 & 0.821 & 0.238 & 4.176 & 236635 \\
        MG.B.4 & 280.11 & 0.003 & 0.432 & 0.130 & 1.706 & 123793 \\
        MG.B.8 & 148.29 & 0.003 & 0.442 & 0.116 & 0.893 & 60600 \\
        LU.A.2 & 2116.54 & 0.002 & 0.110 & 0.030 & 0.532 & 28754 \\
        LU.A.4 & 1102.50 & 0.002 & 0.069 & 0.017 & 0.255 & 14915 \\
        LU.A.8 & 574.47 & 0.003 & 0.067 & 0.016 & 0.192 & 8655 \\
        LU.B.2 & 9712.87 & 0.002 & 0.357 & 0.104 & 1.734 & 101975 \\
        LU.B.4 & 4757.80 & 0.003 & 0.190 & 0.056 & 0.808 & 53522 \\
        LU.B.8 & 2444.05 & 0.004 & 0.222 & 0.057 & 0.548 & 30134 \\
        EP.A.2 & 123.81 & 0.002 & 0.010 & 0.003 & 0.074 & 1834 \\
        EP.A.4 & 61.92 & 0.003 & 0.011 & 0.004 & 0.073 & 1743 \\
        EP.A.8 & 31.06 & 0.004 & 0.017 & 0.005 & 0.073 & 1661 \\
        EP.B.2 & 495.49 & 0.001 & 0.009 & 0.003 & 0.196 & 2011 \\
        EP.B.4 & 247.69 & 0.002 & 0.012 & 0.004 & 0.122 & 1663 \\
        EP.B.8 & 126.74 & 0.003 & 0.017 & 0.005 & 0.083 & 1656 \\
        \bottomrule
\end{longtable}

\qtbr{重要的事说三遍}:浮动体、浮动体、浮动体,在排版图表时,一定要使
用浮动体排版,并用\cs{caption}命令添加题注以实现自动编号,\qtbr{万万不
  可}进行手动编号,否则将会失去\qtb{自动化}功能,从而造成不必要的麻烦!

\section{数字与国际单位}

本模板预加载 \pkg{siunitx} 来格式化文中的内联数字,该宏包有大量可定制的参数,
请务必阅读其文档,并在文档导言部分设置格式。

\begin{itemize}
  \item 旋转角度为 \ang{90}、\ang{270}
  \item 分辨率 \num{1920x1080} 的像素数量约为 \num{2.07e6}
  \item 电脑显示器的像素间距为 \SI{1.8}{\nm}、\SI{180}{\um} 还是 \SI{18}{\mm}?
  \item 重力加速度 $g=\SI{9.8}{\kg\per\square\second}$、
  $g=\SI[inter-unit-product=\ensuremath{{}\cdot{}}]{9.8}{\kg\per\square\second}$,
  亦或 $g=\SI[per-mode=symbol]{9.8}{\kg\per\square\second}$
\end{itemize}

\section{中英文之间空格}

很遗憾,目前 \LaTeX{} 和 \CTeX{} 虽然能处理普通汉字与英文之间的间隔,
但是汉字与宏之间的空格仍然需要手工调整,请务必按以下的规则撰写原稿:
\begin{itemize}
  \item[\ding{51}] 如\autoref{fig:sub2} 所示:\verb|如\autoref{fig:sub2} 所示|,这个宏返回的是\qtmark{图 x-xx},
  所以前面两个汉字之间不能加空格,后面数字与汉字之间必须加空格;
  \item[\ding{51}] 距离为 1.7~个天文单位:\verb|距离为 1.7~个天文单位|,前面可以不加空格(\CTeX 会修正),
  后面必须加 \verb|~| 以防止在 \qtmark{1.7}与\qtmark{个}之间换行。此时更推荐写成 \SI{1.7}{au}:\verb|\SI{1.7}{au}|。
\end{itemize}

%%% Local Variables: 
%%% mode: latex
%%% TeX-master: "../main.tex"
%%% End:

% 本文件是示例论文的一部分
% 论文的主文件位于上级目录的 `main.tex`

\chapter{公式与参考文献}

本章节介绍由 \nwafuthesis{} 提供的特有的宏。

\section{定理环境}

\nwafuthesis{} 提供了三个宏 \cs{nwafutheorem(g|chap|chapu)} 以定义不同编号方法的定理环境。
\begin{enumerate}
  \item \cs{nwafutheoremg} 的编号只有一个数字;
  \item \cs{nwafutheoremchap} 的编号由\qtmark{章节.序号}构成,不同定理环境的编号是独立的,
  它们的数字编号会重复,如\qtmark{\autoref{ex:oneplus}}后面可能出现\qtmark{\autoref{non:dora}};
  \item \cs{nwafutheoremchapu} 的编号也是由\qtmark{章节.序号}构成,
  但它们的数字编号是统一的,同一个数字不会重复出现(仅限用\cs{nwafutheoremchapu}声明的定理环境之间)。
  如\qtmark{\autoref{def:distance}}后面\textbf{不会}出现\qtmark{假设~2.1},但可能出现\qtmark{定义~2.2}或\qtmark{\autoref{assume:fail}};
\end{enumerate}

关于用这三个宏定义定理环境的样例请参阅\qtmark{\file{setup/format.tex}}。

由于学校没有规定计数的编号,所以所有的定理环境应该由作者来决定编号方式,
这也意味着所有的定理环境都要由作者来定义。

顺便一提,在同一章里同时出现两种编号方式的定理环境,很可能造成混乱,
所以请合理安排定理环境的编号方式。

\subsection*{样例}

\begin{definition}[欧几里得距离]
\label{def:distance}
点$\mathbf{p}$与点$\mathbf{q}$的\textbf{欧几里得距离},是连接该两点的线段($\overline{\mathbf{pq}}$)的长度。

在笛卡尔坐标系下,如果 $n$维欧几里得空间下的两个点 $\mathbf{p}=(p_1, p_2, \dots, p_n)$ 与点
$\mathbf{q} = (q_1, q_2, q_3, \dots, q_n)$,那么点$\mathbf{p}$与点$\mathbf{q}$的距离,
或者点$\mathbf{q}$与点$\mathbf{p}$的距离,由以下公式定义:
\begin{align}
\label{equ:1}
d(\mathbf{p},\mathbf{q}) = d(\mathbf{q},\mathbf{p}) & = \sqrt{(q_1-p_1)^2 + (q_2-p_2)^2 + \cdots + (q_n-p_n)^2} \\
\label{equ:2}
& = \sqrt{\sum_{i=1}^n (q_i-p_i)^2}
\end{align}
\end{definition}

\begin{proof}
由\cs{nwafutheorem(g|chap|chapu)}定义的定理环境支持 \cs{autoref},
比如在\autoref{def:distance}中,\autoref{equ:2}是\autoref{equ:1}的简写。

但是 \cs{autoref} 只能在 \cs{ref} 加上前缀,无法加上后缀。
所以上一句话的后半部分,更推荐手工来写标注 “(\ref{equ:2}) 是 (\ref{equ:1}) 的简写”。

定理环境里面可以换行,不过证明与其他定理环境稍有不同,它是单独定义实现的,
因此末尾会有一个 QED 符号。
\end{proof}

\begin{assumption}
\label{assume:fail}
假设本身就不成立
\end{assumption}

\begin{lines}
\label{s1}
例句1
\end{lines}

与图表一样,公式、定理等也需要采用专用的命令或环境进行排版以实现
编号、交叉引用等\qtb{自动化}处理,\qtbr{万万不可}手动编号、引用!

\section{参考文献}
\label{sec:bib}
参考文献的引用采用\qtmark{著者-出版年}制,如:

\subsection{引用方式}
\subsubsection{著者作为引用主语}

文中提及著者,在被引用的著者姓名或外国著者姓氏之后用
圆括号标注文献出版年,可使用\cs{yearcite}、\cs{textcite}
命令或手动模式引用文献,如:

\begin{center}
  \begin{minipage}[h]{0.9\linewidth}
    \begin{texdemov}%[0.5]
赵耀东\yearcite{赵耀东1998--}认为...;
\textcite{赵耀东1998--}认为...;
赵耀东(\cite*{赵耀东1998--})认为...;
赵耀东(\citeyear{赵耀东1998--})认为...;
    \end{texdemov}
  \end{minipage}
\end{center}

\emph{注意}:手动模式使用\cs{cite*}或\cs{citeyear}命令时,需要在两端加上小括
  号,\qtbr{推荐使用}\cs{textcite}命令。

\subsubsection{提及内容未提及著者}

文中只提及所引用的资料内容而未提及著者,则在引文叙述
文字之后用圆括号标注著者姓名或外国著者姓氏和出版年份,在著者
和年份之间空一格,此时可以使用\cs{cite}命令引用文
献,如:

\begin{center}
  \begin{minipage}[h]{0.9\linewidth}    
    \begin{texdemov}%[0.5]      、
孟德尔发现了一个很重要的现象,即红、白花豌豆杂交后的所结种子
第二年长出的植株的红白花比例为3:1\cite{fzx1962}。%
    \end{texdemov}
  \end{minipage}
\end{center}

\subsubsection{同一著者多篇文献}

引用同一著者不同年份出版的多篇文献时,后者只注出版年;
引用同一著者在同一年份出版的多篇文献时,无论正文还是文末,年
份之后用英文小写字母 a、b、c 等加以区别。按年份递增顺序排列,
不同文献之间用逗号隔开。此时可以使用\cs{cite}命令引用文
献,如:

\begin{center}
  \begin{minipage}[h]{0.9\linewidth}    
    \begin{texdemov}%[0.5]      、
      UML基础和Rose建模教程中给出了大量案例及案例分析\cite{蔡敏2006a--,蔡敏2006b--}。%
    \end{texdemov}
  \end{minipage}
\end{center}

\subsubsection{两著者文献}

引用两个著者的文献时,两个著者之间加\qtmark{和}(中文)或
\qtmark{and}(英文)。此时可以使用\cs{cite}命令引用文
献,如:

\begin{center}
  \begin{minipage}[h]{0.9\linewidth}    
    \begin{texdemov}%[0.5]      、
利用基于Matlab的计算机仿真\cite{郭文彬2006--},研究了UWB和窄带通讯中的信号共存特性\cite{Chiani2009-231-254}。%
    \end{texdemov}
  \end{minipage}
\end{center}

\subsubsection{三个以上著者文献}

引用三个以上著者时,只标注第一著者姓名,其后加\qtmark{等}(中
文)或\qtmark{et al.}(英文)。此时可以使用\cs{cite}命令引用文
献,如:

\begin{center}
  \begin{minipage}[h]{0.9\linewidth}    
    \begin{texdemov}%[0.5]      、
      UML基础和Rose建模教程中详细说明了其基本方法和技巧\cite{蔡敏2006--}。你不好好学点\LaTeX{}基本命令还真不行\cite{r9}。%
    \end{texdemov}
  \end{minipage}
\end{center}

\subsubsection{同一处引用多篇文献}

同一处引用多篇文献时,按著者字母顺序排列,不同著者文
献之间用分号隔开。此时可以使用\cs{cite}命令引用文
献,注意用\qtbr{逗号}分开\texttt{citeKey}就好,如:

\begin{center}
  \begin{minipage}[h]{0.9\linewidth}    
    \begin{texdemov}%[0.5]      、
      同时引用多个文献\cite{r2,r3,r4,r6}。%
    \end{texdemov}
  \end{minipage}
\end{center}
  
\subsubsection{多次引用同一著者的同一文献}

多次引用同一著者的同一文献,在正文中标注著者与出版年,
并在\qtmark{()}内以以冒号形式标注引文页码。此时可以使用\cs{parencite}命令引用文
献,注意用可选参数指定引用页码,如:

\begin{center}
  \begin{minipage}[h]{0.9\linewidth}    
    \begin{texdemov}%[0.5]      、
      在文献\parencite[20-22]{程根伟1999-32-36}说了一, 在文献\parencite[55-60]{程根伟1999-32-36}说了二。%
    \end{texdemov}
  \end{minipage}
\end{center}

\subsection{输出参考文献列表}

参考文献列表的输出只需在需要输出文献的位置,使用命令\cs{printbibliography}进行输出即可。

\subsection{参考文献数据文件准备}

\LaTeX 文档中生成参考文献一般都需要准备一个参考文献数据源文件即
\qtmark{*.bib}文件。这一文件内保存有各条参考文献的信息,具体可以参考
biblatex宏包手册和biblatex-gb7714-2015样式包手册\cite{胡振震2019}中关
于域信息录入的说明。

参考文献源文件本质上只是一个文本文件,只是其内容需要遵守BibTeX格式,参
考文献源文件可以有多种生成方式,具体可参考\LaTeX{}文档中文参考文献的
biblatex 解决方案\parencite[2.2节]{胡振震2016}。

\note{学校的参考文献格式并不完全符合\texttt{GB7714-2015}参考文献著录标准,强烈建议学校参考文
  献执行\texttt{GB7714-2015}参考文献著录标准!}

本模板采用由胡振震维护的
\qtmark{符合 GB/T 7714-2015 标准的 biblatex 参考文献样式}实现参考文献
的编排\cite{胡振震2019},其Github链接为
\url{https://github.com/hushidong/biblatex-gb7714-2015}。
大家也可以通过\TeX~Live的 \verb|texdoc gb7714-2015|命令查看其使用说明。

关于著者-出版年样式命令的详细说明可参见胡振震\qtmark{符合 GB/T
  7714-2015 标准的 biblatex 参考文献样式}说明中的中的相关内容
\parencite[2.2、2.3节]{胡振震2016}。

\qtbr{切记:}与图表、公式、定理等一样,请使用专用命令引用并输出参考文
献,以实现参考文献的\qtb{自动化}处理,\qtbr{万万不可}手动编写参考文献!


%%% Local Variables: 
%%% mode: latex
%%% TeX-master: "../main.tex"
%%% End:

% 本文件是示例论文的一部分
% 论文的主文件位于上级目录的 `main.tex`

\chapter{多级标题}

\section{演示一级标题}
\subsection{演示二级标题}
\subsubsection{演示三级标题}

\section{使用定理环境}

使用 \cs{nwafutheoremchap} 定义的定理环境,其数字编号是可以重复的。

\begin{nonsense}
\label{non:dora}
哆啦A梦写的论文被拒稿的可能性很高
\footnote{出处:\url{https://www.math.kyoto-u.ac.jp/~arai/latex/presen2.pdf} 的最后一页}。
\end{nonsense}

\begin{exercise}
\label{ex:oneplus}
证明$1+1 = 2$。
\footnote{Testing footnote with English spaces}
\end{exercise}

\begin{nonsense}[右边的胡诌是真的]
“练习”与“胡诌”定理环境的编号是相互独立的,它们的数字编号允许重复,
如“\autoref{non:dora}”和“\autoref{ex:oneplus}”。
\end{nonsense}

\begin{exercise}
按照本文所演示的方法,利用 \cs{nwafutheorem(g|chap|chapu)} 来定义您的论文中所需要的定理环境。
\end{exercise}

\begin{lines}
\label{s2}
例句2
\end{lines}

\autoref{s2} 没有章节编号,它是全局编号的,它可以用在外国系论文中来枚举例句。

%%% Local Variables: 
%%% mode: latex
%%% TeX-master: "../main.tex"
%%% End:

% 本文件是示例论文的一部分
% 论文的主文件位于上级目录的 `main.tex`

\chapter{结论与展望}

结个论,展个望。

\section{结果}
用\LaTeX 写论文还是蛮轻松的。

\section{展望}
以后还要设计更多,更方便的命令来实现高效\LaTeX 论文撰写。



%%% Local Variables: 
%%% mode: latex
%%% TeX-master: "../main.tex"
%%% End:


% 排版参考文献表
\bibliomatter
% \nocite{*}
\nocite{广西壮族自治区林业厅1993--,r4,张若凌2004--,于潇2012-1518-1523,马克思2013-302-302,张田勤2000--,萧钰2001--,刘加林1993--,张志祥1998--,n42,n43}
\printbibliography[heading=bibintoc]%

% 排版附录(推荐用appendix环境排版)
% \appendix
% 如果需要附录的话,在这里 include
% \include{content/05ex_duplicate}
% \include{content/06ex_postscript}
\begin{appendix}
  % 本文件是示例论文的一部分
% 论文的主文件位于上级目录的 `main.tex`

\chapter{查重和其他注意事项}

\section{查重}

先说结论:{\large\qtbr{知网完全支持pdf查重}},学校学院也接收pdf格式的论文,这个无需担心。

如果导师只接受Word版论文,那也就没有办法了,你就用Word吧,只要下点功夫,也不是个事。建议大家提前和指导老师进行沟通,以确认能不能提交pdf格式论文。

\section{批注}
在论文撰写过程中,pdf格式的论文,批注是一个问题,如果对\LaTeX 和基于Git的版本管理并不了解,就只能使用Adobe Acrobat、平板手写等软件,对pdf文件本身进行批注,相比于word确实有些麻烦。

强烈推荐使用Git\footnote{\url{https://git-scm.com/}}、Beyond Compare\footnote{\url{https://www.scootersoftware.com/}}等工具,辅以\LaTeX 本身的注释进行批注以及版本管理,非常清晰直观,操作也简单。

\section{毕业设计与毕业论文的区别}
这里特别对使用本模板的本科同学们做出提醒,请查看毕业设计基本信息中的毕设类别,共有两类:\qtb{毕业设计}和\qtb{毕业论文}。因此在\verb!\documentclass[]{nwafuthesis}!的选项中需要标明\textbf{Design}(毕业设计)或者\textbf{Paper}(毕业论文),使论文使用正确的封面和独创性声明。

\section{单面打印\& 双面打印}
学校并没有规定论文打印的方式,考虑到部分同学有双面打印的需求,可以在文档选项中使用oneside/twoside来切换单面打印和双面打印。

\section{封面打印\& 装订}
建议大家去指定打印部门打印封面并装订,以免打印装订不合格。

  \chapter{后记}

\section{吐槽}

\verb!\begin{轻松+愉快}!

做模板过程中遇到的大问题,在于如何正确理解学校对论文格式的要求。
虽然有《本科毕业设计(论文)撰写格式要求》、《研究生学位论文撰写要求》,
但这些要求依然不够细致,因为那些要求都是假定你用 Word 来写论文的,要求里的内容是 Word 设置的操作方法,
所以还要先学习 Word 的排版算法,因此,本模板
但还有很多细节部分,因为能力有限,没能实现。

最后容我吐槽一下学校的 Word 模板,那个 Word 模板可能从最初做出来后,就基本没有变化。
很多编号的事情都要由手工来完成,比如说目录页码、
各级标题的编号、题注等。这些完全可以自动编号的工作,如果要手工做的话是非常累人和容易出错的。

同时,强烈建议学校能用标准的地方一定要用标准,比如参考文献的GB7714-2015标准!

\section{明天}

转眼间n年过去,又到了写毕业论文的时候了,一直想完成我们学校的毕业论文模板,今天总算有了一个初稿。

目前, \nwafuthesis{} 应该还有相当多的问题,但没有用户的话,由于作者能力有限,很难发现这些问题,
还请各位使用 \nwafuthesis{} 的先行者们(Pioneers) 能及时反馈意见和建议。

愿所有使用 \nwafuthesis{} 的人,不会被评审老师指责格式问题。

\end{appendix}

% 排版致谢
\backmatter
\makeatletter % necessary for using \@chapapp
\renewcommand{\chaptermark}[1]{%
  \markboth{#1}{}}
\makeatother
\chapter[致谢]{致\hskip\ccwd{}谢}

在此感谢对本论文作成有所帮助的人。


%%% Local Variables: 
%%% mode: latex
%%% TeX-master: "../main.tex"
%%% End:


\end{document}

%%% Local Variables:
%%% mode: latex
%%% TeX-master: t
%%% End:
